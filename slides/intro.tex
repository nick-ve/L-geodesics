\Transcb{yellow}{blue}{Introduction}
\begin{itemize}
\item Relativity provides a consistent framework to describe
\begin{itemize}
\item High-Energy phenomena
\item Gravity
\end{itemize}
\item[$\ast$] {\blue But how can we make predictions that can be tested experimentally ?}
\item[] For example : The trajectory of a test body in curved space-time
\item Test body :
\item[] Object with negligible mass that does not significantly curve space-time itself
\item[] For instance a satellite orbiting the Earth
\item We need somehow to derive the equations governing the motion of test bodies
\item[] Obviously Newtonian mechanics won't work
\item[] We will limit ourselves to space-time curvature only (no Electric forces etc.)
\item[] $\rightarrow$ We will see that a Hamilton-Lagrange approach can be applied
\item But first we will investigate a relatively simple measurement
\item[] which is in fact applied to correct timing via the GPS system
\end{itemize}

\Tr
{\red
\begin{center}
Exercise
\end{center}
%
\begin{itemize}
\item Consider a satellite moving eastwards over the Earth equator
      at a height $h$ above the Earth's surface with constant velocity $v$ w.r.t. the ground
\item[] The satellite contains a clock which measures the satellite's proper time $\tau_{s}$
\item[] In a ground station at the equator an identical clock measures the proper time $\tau_{g}$ of the station
\item Consider the Earth as a perfect sphere with mass $M$, radius $R$ and Schwarzschild radius $R_{s}$
\item The Earth rotates around its axis with a constant angular velocity $\Omega$
\item When the satellite passes over the ground station, both clocks are synchronised
\item When the satellite passes over the ground station after one circumnavigation,
      the elapsed proper times $\Delta \tau_{s}$ and $\Delta \tau_{g}$ are compared
\item[$\ast$] By assuming that all velocities are non-relativistic, show that
\item[] $\displaystyle \left(\frac{\Delta \tau_{s}}{\Delta \tau_{g}}\right)^{2}=
         \frac{1-R_{s}/(R+h)-([(R+h)\Omega+v]/c)^{2}}{1-R_{s}/R-(R\Omega/c)^{2}}$
\end{itemize}
}