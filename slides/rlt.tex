\Transcb{yellow}{blue}{Going relativistic}
\begin{itemize}
\item In relativity everything is determined via the metric
\item[] Special relativity : $\d s^{2}=\eta_{\mu \nu}\, \d x^{\mu}\, \d x^{\nu}$
\item[] General relativity : $\d s^{2}=g_{\mu \nu}(\Fvec{x})\, \d x^{\mu}\, \d x^{\nu}$
\item[] where $\d\Fvec{x}=(c\,\d t,\d\vec{r})$ and $\eta_{\mu\nu}$=diag(1,-1,-1,-1)
\item Consider two events $A$ and $B$ in space-time $\rightarrow \displaystyle \Delta s=\int_{A}^{B} \d s$
\item[] where the integral follows some path (i.e. worldline) in space-time
\item[$\ast$] {\blue Hey, this looks like the classical Hamilton-Lagrange path integral}
\item[] $\displaystyle {\cal P}=\int_{t_{1}}^{t_{2}} L\, \d t$ where $L=L(q_{k},\dot{q}_{k})$ and $q_{k}=q_{k}(t)$
\item Consider $A$ and $B$ as two fixed endpoints of some worldline in space-time
\item[] Any worldline between $A$ and $B$ can be parametrized as $x^{\mu}=x^{\mu}(\lambda)$
\item[] which yields : $\displaystyle \d x^{\mu}=\frac{\partial x^{\mu}}{\partial \lambda}\, \d \lambda=
        \frac{\d x^{\mu}}{\d \lambda}\,\d \lambda$
\end{itemize}

\Tr
\begin{itemize}
\item This then leads to the following expressions :
\item[] Special relativity : $\displaystyle \Delta s=\int_{\lambda_{1}}^{\lambda_{2}}
        \sqrt{\eta_{\mu \nu}\, \frac{\d x^{\mu}}{\d \lambda}\, \frac{\d x^{\nu}}{\d \lambda}}\,\, \d \lambda$
\item[] General relativity : $\displaystyle \Delta s=\int_{\lambda_{1}}^{\lambda_{2}}
        \sqrt{g_{\mu \nu}(\Fvec{x})\, \frac{\d x^{\mu}}{\d \lambda}\, \frac{\d x^{\nu}}{\d \lambda}}\,\, \d \lambda$
\item[] which can be written as {\blue $\displaystyle \Delta s=\int_{\lambda_{1}}^{\lambda_{2}} L\, \d \lambda$}
      with {\blue $\displaystyle L=L\left(x^{\mu},\frac{\d x^{\mu}}{\d \lambda} \right)$}
\item[$\ast$] Compare this with $\displaystyle {\cal P}=\int_{t_{1}}^{t_{2}} L\, \d t$ with $L=L(q_{k},\dot{q}_{k})$
\item {\blue If the actual worldline (called {\red Geodesic}) corresponds to an extremum of $\Delta s$}\\
      we obtain the {\blue Relativistic Lagrange equations}
\end{itemize}
%
\begin{center}
{\red \shabox{$\displaystyle \frac{\d}{\d \lambda}\left(\frac{\partial L}{\partial(\d x^{\mu}/\d \lambda)}\right)=
               \frac{\partial L}{\partial x^{\mu}}$}}
\end{center}
